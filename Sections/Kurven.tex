\section{Kurven}
\subsection{Differenzation}
Die Ableitung $\vec{f}'(t)$ von $\vec{f}(t)$ wird durch Ableiten der einzelnen Komponenten berechnet.
\[
\vec{f}'(t) = \begin{pmatrix}
	x'(t) \\ y'(t)
\end{pmatrix}
\]

~\\
\textbf{Flächeninhalt}:
Eine \textbf{kreuzungsfreie, geschlossene} Kurve $\vec{f}(t) = \begin{pmatrix}
	x(t) \\ y(t)
\end{pmatrix}$ mit $t \in [t_1;t_2]$ hat eine Fläche von
\[
F = \left|\int_{t_1}^{t_2}y(t)\cdot x'(t)dt\right| = \left|\int_{t_1}^{t_2}x(t)\cdot y'(t)dt\right|\]


\subsection{Integration}
Im Allgemeinen kann eine Kurve integriert werden mit $r \leq t \leq s$:
\[
I = \int_{k}f\cdot ds = \int_{k}\vec{f}(\vec{x})\left|d\vec{x}\right| \xRightarrow[\text{Erweitern mit } dt]{!} \int_{r}^{s}f(\vec{x}(t))\cdot \left|\vec{x}'(t)\right|dt
\]
~\\
\textbf{Kurvenlänge}:
Die Länge $l$ einer Kurve mit Parametergleichung $\vec{f}(t)$ mit $t \in [t_1;t_2]$ beträgt:
\[
l = \int_{t_1}^{t_2}\left|\vec{f}'(t)\right|dt = \int_{t_1}^{t_2}\sqrt{[x'(t)]^2 + [y'(t)]^2 + \cdots}dt
\]

\subsection{Übersicht}
\begin{center}
	\rotatebox{90}{
		\bgroup
		\def\arraystretch{2.5}
		\begin{tabular}[h]{l|c|c|c}
			& \textbf{Kart.} & \textbf{Para.} & \textbf{Polar} \\
			\toprule
			Steigung & $f'(x)$ & $\frac{\dot y}{\dot x}$ & $\frac{r'\sin \varphi + r \cos\varphi}{r'\cos \varphi - r\sin\varphi} $ \\
			\hline
			Bogenlänge & $\int_{x_0}^{x_1}\sqrt{1 + f'(x)^2}dx$ & $\int_{t_0}^{t_1}\sqrt{\dot x^2 + \dot y^2}dt$ & $\int_{\varphi_0}^{\varphi_1}\sqrt{r'^2 + r^2}d\varphi$ \\
			\hline
			Krümmung & $\frac{f''(x)}{\sqrt{1 + f'(x)^2}^3}$ & $\frac{\ddot y \dot x - \dot y \ddot x}{\sqrt{\dot x^2 + \dot y^2}^3}$ & $\frac{2\dot r^2 - r \ddot r + r^2}{\sqrt{\dot r^2 + r^2}^3}$ \\
			\hline
			Fläche & $\int_{a}^{b}f(x)dx$ & $\frac{1}{2}\int_{t_0}^{t_1}[x\dot y - \dot xy]dt$ & $\frac{1}{2}\int_{\varphi_0}^{\varphi_1} r^2 d\varphi$\\
			\hline
			Volumen um $\vec{x}^*$ & $\pi \int_{a}^{b}f(x)^2dx$ & $\left|\pi \int_{t_0}^{t_1}y^2 \cdot \dot xdt\right|$ & $\left|\pi \int_{\varphi_0}^{\varphi_1}r^2\sin^2\varphi \cdot (\dot r \cos\varphi - r\sin\varphi)d\varphi\right|$ \\
			\hline
			Oberfläche um $\vec{x}^*$ & $2\pi \int_{a}^{b}\left|f(x)\right|\cdot\sqrt{1 + f'(x)^2}dx$ & $\left|2\pi\int_{t_0}^{t_1}\left|y\right|\cdot\sqrt{\dot x^2 + \dot y^2}dt\right|$ & $\left|2\pi \int_{\varphi_0}^{\varphi_1}\left|r\sin\varphi\right|\cdot\sqrt{r^2 + \cdot r^2}d\varphi\right|$
		\end{tabular}
		\egroup
	}
\end{center}
\noindent $^*$) Für Rotation um $\vec{y}$ können in Para. die Funktionen $y$ durch $x$ ersetzt werden. Für Kart. kann die Umkehrfunktion $f^{-1}(x)$ verwendet werden.\\\textbf{Achtung}: Integral Grenzen müssen in Kart. auch angepasst sein:
\begin{align*}
	f(x) = \frac{1}{3}x^2 &\xRightarrow[]{x \leftrightarrow y} f^{-1}(y) = \sqrt{3y} \\
	\int_{a}^{b} (\cdots) dx  &\xRightarrow[]{} \int_{f^{-1}(a)}^{f^{-1}(b)} (\cdots) dy
\end{align*}
